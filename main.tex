% !Mode:: "TeX:UTF-8"
%!TEX program  = xelatex

\documentclass{cumcmthesis}
%\documentclass[withoutpreface,bwprint]{cumcmthesis} %去掉封面与编号页

\usepackage{url}
\title{全国大学生数学建模竞赛编写的 \LaTeX{} 模板}
\tihao{A}
\baominghao{6799}
\schoolname{同济大学}
\membera{陈成}
\memberb{许奕}
\memberc{王梓聿}
\supervisor{}
\yearinput{2025}
\monthinput{08}
\dayinput{22}
\begin{document}

\maketitle

\begin{abstract}
cumcmthesis 是为全国大学生数学建模竞赛编写的\LaTeX{}模板, 旨在让大家专注于 论文的内容写作, 而不用花费过多精力在格式的定制
和调整上. 本手册是相应的参考, 其中提供了一些环境和命令可以让模板的使用更为方便. 同时需要注意, 使用者需要有一定的 \LaTeX{} 
使用经验, 至少要会使用常用宏包的一些功能, 比如参考文献, 数学公式, 图片使用, 列表环境等等. 例子文件参看 example.pdf.

\keywords{折叠桌\quad  曲线拟合\quad   非线性优化模型\quad  受力分析}
\end{abstract}

%目录
% \tableofcontents

\section{问题重述}

创意平板折叠桌注重于表达木制品的优雅和设计师所想要强调的自动化与功能性。为了增大有效使用面积。设计师以长方形木板的宽为
直径截取了一个圆形作为桌面,又将木板剩余的面积切割成了若干个长短不一的木条,
每根木条的长度为平板宽到圆上一点的距离,分别用两根钢筋贯穿两侧的木条,使用者只需提起木板的两侧,便可以在重力的作用下达
到自动升起的效果,相互对称的木条宛如下垂的桌布,精密的制作工艺配以质朴的木材,
让这件工艺品看起来就像是工业革命时期的机器。

\subsection{问题的提出}

围绕创意平板折叠桌的动态变化过程、设计加工参数,本文依次提出如下问题:

(1)给定长方形平板尺寸 ($120 cm \times 50 cm \times 3 cm$),每根木条宽度(2.5 cm),连接桌腿木条的钢筋的位置,折叠
后桌子的高度(53 cm)。要求建立模型描述此折叠桌的动态变化过程,并在此
基础上给出此折叠桌的设计加工参数和桌脚边缘线的数学描述。

(2)......


\section{问题分析}

\subsection{问题一分析}
题目要求建立模型描述折叠桌的动态变化图,由于在折叠时用力大小的不同,我们不能描述在某一时刻折叠桌的具体形态,但我们可以用每
根木条的角度变化来描述折叠桌的
动态变化。首先,我们知道折叠桌前后左右对称,我们可以运用几何知识求出四分之一木条的角度变化。最后,根据初始时刻和最终形态
两种状态求出桌腿木条开槽的长度。

问题流程图:
% \begin{figure}[!h]
% \centering
% \includegraphics[width=.6\textwidth]{1.png}
% \caption{问题三流程图}
% \end{figure}


\section{模型假设}

\begin{itemize}
\item 忽略实际加工误差对设计的影响;
\item 木条与圆桌面之间的交接处缝隙较小,可忽略;
\item 钢筋强度足够大,不弯曲;
\item 假设地面平整。
\end{itemize}

\section{符号说明}
\begin{center}
\begin{tabular}{cc}
 \hline
 \makebox[0.3\textwidth][c]{符号}	&  \makebox[0.4\textwidth][c]{意义} \\ \hline
 D	    & 木条宽度(cm) \\ \hline
 L	    & 木板长度(cm)  \\ \hline
 W	    & 木板宽度(cm)  \\ \hline
 N	    & 第n根木条  \\ \hline
 T	    & 木条根数  \\ \hline
\end{tabular}
\end{center}


\section{问题一模型建立及求解}


\section{问题二模型建立及求解}


\section{问题三模型建立及求解}


\section{问题四模型建立及求解}


\section{模型检验及结果分析}


\section{模型评价及推广}


%参考文献
\begin{thebibliography}{9}%宽度9
 \bibitem{bib:one} ....
 \bibitem{bib:two} ....
\end{thebibliography}

\newpage
%附录
\begin{appendices}
\section{python源程序}
\begin{lstlisting}[language=python]
import numpy as np
import matplotlib.pyplot as plt
import math
def draw_table(length, width, height, bar_width, bar_count):
    # 计算每根木条的角度
    angles = np.linspace(0, np.pi / 2, bar_count)
    
    # 绘制桌面
    plt.plot([0, length], [height, height], 'k-', linewidth=2)
    
    # 绘制木条
    for i in range(bar_count):
        angle = angles[i]
        x = np.linspace(0, length, 100)
        y = height - (bar_width / 2) * np.tan(angle) * (x / length)
        plt.plot(x, y, 'b-', linewidth=1.5)
    
    plt.xlim(-10, length + 10)
    plt.ylim(-10, height + 10)
    plt.gca().set_aspect('equal', adjustable='box')
    plt.title('折叠桌动态变化示意图')
    plt.xlabel('长度 (cm)')
    plt.ylabel('高度 (cm)')
    plt.grid()
    plt.show()

 \end{lstlisting}
\end{appendices}

\end{document}