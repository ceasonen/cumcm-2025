%%%%%%%%%%%%%%%%%%%%%%%%%%%%%%%%%%%%%%%%%%%%%%%%%%%%%%%%%%%%%%%%%%%%%%%%%%%%%%%
% 全国大学生数学建模竞赛 - AI工具使用详情报告 LaTeX 模板
% 根据《全国大学生数学建模竞赛人工智能工具使用规定(2025年试行)》制作
% 请注意,最终提交的文件应为PDF格式,并以“AI工具使用详情”为文件名。 
%%%%%%%%%%%%%%%%%%%%%%%%%%%%%%%%%%%%%%%%%%%%%%%%%%%%%%%%%%%%%%%%%%%%%%%%%%%%%%%

\documentclass[UTF8]{ctexart}

% --- 宏包设置 ---
\usepackage{geometry}      % 设置页面边距
\usepackage{amsmath}       % 提供数学环境
\usepackage{graphicx}      % 用于插入图片
\usepackage[
    colorlinks=true,       % 彩色链接
    linkcolor=black,       % 目录、内部链接颜色
    anchorcolor=black,     % 锚点颜色
    citecolor=black,       % 参考文献引用颜色
    urlcolor=blue          % URL颜色
]{hyperref}                % 超链接功能
\usepackage{listings}      % 用于插入代码或文本块
\usepackage{xcolor}        % 提供颜色定义

% --- 页面格式 ---
\geometry{a4paper, left=2.5cm, right=2.5cm, top=2.5cm, bottom=2.5cm} % 设置A4页面及边距

% --- listings 环境定义 ---
% 用于美观地展示Prompt和AI回复
\definecolor{codegreen}{rgb}{0,0.6,0}
\definecolor{codegray}{rgb}{0.5,0.5,0.5}
\definecolor{codepurple}{rgb}{0.58,0,0.82}
\definecolor{backcolour}{rgb}{0.95,0.95,0.92}

\lstdefinestyle{mystyle}{
    backgroundcolor=\color{backcolour},
    commentstyle=\color{codegreen},
    keywordstyle=\color{magenta},
    numberstyle=\tiny\color{codegray},
    stringstyle=\color{codepurple},
    basicstyle=\ttfamily\footnotesize,
    breakatwhitespace=false,
    breaklines=true,
    captionpos=b,
    keepspaces=true,
    numbers=left,
    numbersep=5pt,
    showspaces=false,
    showstringspaces=false,
    showtabs=false,
    tabsize=2,
    frame=single,
    rulecolor=\color{black},
    title=\lstname
}
\lstset{style=mystyle}

% --- 文档信息 ---
\title{\heiti AI工具使用详情}
\author{队伍编号: \underline{\qquad\qquad} \\ \vspace{0.5cm} 队员: \underline{\qquad\qquad}, \underline{\qquad\qquad}, \underline{\qquad\qquad}}
\date{\today}

%%%%%%%%%%%%%%%%%%%%%%%%%%%%%%%%%%%%%%%%%%%%%%%%%%%%%%%%%%%%%%%%%%%%%%%%%%%%%%%
% --- 文档正文开始 ---
%%%%%%%%%%%%%%%%%%%%%%%%%%%%%%%%%%%%%%%%%%%%%%%%%%%%%%%%%%%%%%%%%%%%%%%%%%%%%%%
\begin{document}

\maketitle
\thispagestyle{empty} % 封面不显示页码
\newpage

\tableofcontents % 自动生成目录
\newpage

% -----------------------------------------------------------------------------
% 第一部分:所用AI工具名称和版本
% 来源:《规定》第4条第(2)款 
% -----------------------------------------------------------------------------
\section{所用AI工具名称和版本}

\begin{itemize}
    \item \textbf{工具一:}
          \begin{itemize}
              \item \textbf{名称:} DeepSeek
              \item \textbf{版本/型号:} DeepSeek-R1-0528
              \item \textbf{开发机构/公司:} 深度求索 (DeepSeek)
              \item \textbf{使用日期:} 2025-09-05
          \end{itemize}
          \vspace{0.3cm} % 增加条目间距
    \item \textbf{工具二:}
          \begin{itemize}
              \item \textbf{名称:} GitHub Copilot
              \item \textbf{版本/型号:} 2.10.1
              \item \textbf{开发机构/公司:} GitHub / OpenAI
              \item \textbf{使用日期:} 2025-09-06
          \end{itemize}
    % 如果有更多工具,请在此处继续添加
\end{itemize}


% -----------------------------------------------------------------------------
% 第二部分:具体使用目的和环节
% 来源:《规定》第4条第(2)款 
% -----------------------------------------------------------------------------
\section{具体使用目的和环节}

\subsection{DeepSeek (大语言模型)}
\begin{itemize}
    \item \textbf{使用环节:} 问题分析与论文写作阶段。
    \item \textbf{使用目的:}
    \begin{enumerate}
        \item 对某些专业术语或背景知识进行快速检索和理解。
        \item 在论文写作中,对部分中文表述进行润色,使其语言更流畅、更具学术性。
        \item 辅助生成部分流程图的Mermaid代码。
    \end{enumerate}
\end{itemize}

\subsection{GitHub Copilot (代码辅助工具)}
\begin{itemize}
    \item \textbf{使用环节:} 数据处理与模型编程实现阶段。
    \item \textbf{使用目的:}
    \begin{enumerate}
        \item 辅助编写数据预处理的 Python 脚本,如数据清洗、缺失值填充等。
        \item 根据函数功能注释,自动生成函数代码框架。
        \item 对已有代码进行快速调试和优化建议。
    \end{enumerate}
\end{itemize}


% -----------------------------------------------------------------------------
% 第三部分:关键交互记录 (重要提示词与回复)
% 来源:《规定》第4条第(2)款 
% -----------------------------------------------------------------------------
\section{关键交互记录 (重要提示词与回复)}

\subsection{交互一: 使用 DeepSeek 进行文本润色}

\subsubsection*{重要提示词 (Prompt)}
\begin{lstlisting}[language=text, title={用户输入}]
请将以下这段关于模型假设的描述改写得更专业、更严谨:
“我们认为市场的需求量是基本不变的,而且天气好坏对我们的模型没啥大影响,为了方便计算,我们就没考虑天气。”
\end{lstlisting}

\subsubsection*{AI 回复 (Response)}
\begin{lstlisting}[language=text, title={DeepSeek 回复}]
为确保模型的简洁性与分析的有效性,我们提出以下两条核心假设:
1.  **市场需求稳定性假设:** 在本模型的研究周期内,我们假设市场总需求量保持稳定。此假设旨在剥离宏观市场波动对核心变量的干扰,从而更精确地评估模型内部参数的相互作用关系。
2.  **气象因素的独立性假设:** 经初步分析,气象条件(如温度、降雨量等)与本模型研究的关键指标关联性较弱。因此,为降低模型复杂度并聚焦于主要矛盾,本研究暂不将气象因素纳入考量范围。
\end{lstlisting}

\subsection{交互二: 使用 GitHub Copilot 生成代码}

\subsubsection*{重要提示词 (Prompt)}
在 Python 代码编辑器中,输入以下注释和函数签名:
\begin{lstlisting}[language=Python, title={用户输入 (注释)}]
# Function to calculate the moving average of a time series
# data: list of numbers
# window_size: integer, the size of the moving window
def moving_average(data, window_size):
\end{lstlisting}

\subsubsection*{AI 回复 (Response)}
\begin{lstlisting}[language=Python, title={GitHub Copilot 自动补全代码}]
# Function to calculate the moving average of a time series
# data: list of numbers
# window_size: integer, the size of the moving window
def moving_average(data, window_size):
    """Calculates the moving average of a list."""
    if window_size <= 0 or window_size > len(data):
        raise ValueError("Window size must be positive and not exceed data length.")
    
    averages = []
    for i in range(len(data) - window_size + 1):
        window = data[i:i + window_size]
        window_average = sum(window) / window_size
        averages.append(window_average)
    return averages
\end{lstlisting}


% -----------------------------------------------------------------------------
% 第四部分:采纳和人工修改情况
% 来源:《规定》第4条第(2)款 
% -----------------------------------------------------------------------------
\section{采纳和人工修改情况}

\subsection{针对交互一 (DeepSeek 文本润色)}
\begin{itemize}
    \item \textbf{采纳情况:} 完全采纳了AI生成的文本框架和专业术语。
    \item \textbf{人工修改:}
    \begin{enumerate}
        \item 将“核心假设”修改为“基本假设”,以符合我们论文的整体用词风格。
        \item 在第二点假设的描述中,补充了具体的分析依据,将“关联性较弱”修改为“经斯皮尔曼相关性检验,$p > 0.1$,关联性不显著”。
    \end{enumerate}
\end{itemize}

\subsection{针对交互二 (GitHub Copilot 代码生成)}
\begin{itemize}
    \item \textbf{采纳情况:} 采纳了核心的循环和切片计算逻辑。
    \item \textbf{人工修改:}
    \begin{enumerate}
        \item 增加了对输入 `data` 非空和元素类型为数字的检查,使函数更健壮。
        \item 使用 `numpy` 库进行了代码重构,以提高大规模数据处理时的计算效率。修改后的核心代码为: \\
        \texttt{averages = np.convolve(data, np.ones(window\_size), 'valid') / window\_size}。
    \end{enumerate}
\end{itemize}

\end{document}
%%%%%%%%%%%%%%%%%%%%%%%%%%%%%%%%%%%%%%%%%%%%%%%%%%%%%%%%%%%%%%%%%%%%%%%%%%%%%%%
% --- 文档正文结束 ---
%%%%%%%%%%%%%%%%%%%%%%%%%%%%%%%%%%%%%%%%%%%%%%%%%%%%%%%%%%%%%%%%%%%%%%%%%%%%%%%